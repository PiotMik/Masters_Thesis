\begin{wstep}[Wprowadzenie]
	Celem tej pracy jest zbadanie skuteczności modeli \emph{Vine Copula} w zastosowaniu do modelowania spreadu na różnicę cen 3 aktywów. Podajemy teorię niezbędną do zdefiniowania tych modeli, omawiamy obecne podejścia do modelowania spreadu i kalibrujemy model \emph{Vine Copula} do danych rzeczywistych dotyczących \emph{soybean crush spreadu}. W tym modelu numerycznie wyceniamy europejskie opcje na spread.\\
	
	W pierwszym rozdziale wprowadzamy istotne dla teorii kopuł obiekty z zakresu probabilistyki i statystyki. Przedstawiamy wielowymiarowe zmienne losowe, z naciskiem na modele ich rozkładów najczęściej stosowane w praktyce, oraz wskazując na ich istotne ograniczenia. Definiujemy również miary współzależności zmiennych losowych, wychodząc ze statystycznego punktu odniesienia.\\
	
	Drugi rozdział stanowi opis dwuwymiarowych modeli kopułowych i ich rozszerzenia do \emph{Vine Copulas}. Definiujemy w nim dwuwymiarowe kopuły, podajemy przykłady i pokazujemy że są rozszerzeniem rozkładów wielowymiarowych z rozdziału pierwszego. Redefiniujemy przy tym również miary współzależności z rodziału pierwszego, tym razem w języku kopuł.\\
	
	W rozdziale trzecim skupiamy się na pojęciu spreadu aktywów, podajemy ich przykłady obecne na rynkach finansowych i wymieniamy podejścia do ich modelowania. Skupiamy się na podejściu kopułowym, łączącym modelowanie szeregów czasowych z wielowymiarowymi kopułami - w szczególności ze strukturami \emph{Vine Copulas}.\\
	
	Ostatni rozdział prezentuje wynik kalibracji modelu do rzeczywistych danych, tj. cen \emph{soybean crush spread}, czyli różnicy między ceną surowca (ziaren soi), a powstających z niej produktów: mączki sojowej i olejku sojowego. Pokazujemy jak w numeryczny sposób można wycenić w tym modelu opcje europejskie, oraz porównujemy wyniki symulacji modelu używającego \emph{Vine Copula}, a najprostszego modelu wielowymiarowej kopuły gaussowskiej.
	
\end{wstep}
