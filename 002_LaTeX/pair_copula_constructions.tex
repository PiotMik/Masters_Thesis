Zaczniemy od przykładu w trzech wymiarach, aby zilustrować czym jest idea \emph{Pair copula construction} (PCC). Ogólną ideą jest to, aby wielowymiarowy rozkład zmiennej losowej rozbić na dwuwymiarowe komponenty modelowane za pomocą dwuwymiarowych kopuł.\\

W poniższych rozważaniach przydadzą nam się następujące lematy dotyczące relacji między gęstością kopuły a rozkładami warunkowymi:

\begin{lemma}
	Rozkład warunkowy dwuwymiarowej zmiennej losowej można przedstawić w języku kopuły:	
	$$ f_{1|2}(x_1|x_2) =c_{12}(F_1(x_1), F_2(x_2))f_2(x_2).$$
	\label{lem:copula_representation_of_conditional_density}
\end{lemma}
\begin{proof}
	Posługując się twierdzeniem Sklara \ref{thm:sklar_theorem_density} mamy:
	\begin{equation*}
		\begin{split}
			f_{1|2}(x_1, x_2) & = \frac{f_{12}(x_1, x_2)}{f_2(x_2)} \\
			& = \frac{c_{12}(F_1(x_1), F_2(x_2))f_1(x_1)f_2(x_2)}{f_2(x_2)}\\
			& = c_{12}(F_1(x_1), F_2(x_2))f_1(x_1)
		\end{split}
	\end{equation*}
\end{proof}

\subsubsection{Przykład ilustrujący}
\label{subsub:przyklad_3_wymiary}
Aby zilustrować PCC, rozpatrzmy rozkład $3$-wymiarowy o gęstości $f(x_1, x_2, x_3)$. Tę gęstość można przedstawić w postaci:

$$ f(x_1, x_2, x_3) = f_{3|12}(x_3|x_1, x_2)f_{2|1}(x_2|x_1)f_1(x_1).$$

Celem jest sfaktoryzowanie tego wyrażenia do postaci wykorzystującej co najwyżej dwuwymiarowe kopuły lub jednowymiarowe rozkłady. Widzimy, że w tym celu należy rozłożyć $f_{3|12}$ oraz $f_{2|1}$. \\
Korzystając z lematu \ref{lem:copula_representation_of_conditional_density} otrzymujemy bezpośrednio $f_{2|1}$ oraz $f_{3|1}$ (które przyda się za chwilę) jako:
\begin{equation*}
	\begin{split}
		f_{1|2}(x_1|x_2) &= c_{12}(F_1(x_1), F_2(x_2))f_2(x_2),\\
		f_{3|2}(x_3|x_2) &= c_{32}(F_3(x_3), F_2(x_2))f_2(x_2)
	\end{split}
\end{equation*}

Natomiast w celu otrzymania $f_{3|12}$ będziemy musieli przejść przez dwuwymiarowy rozkład $f_{13|2}(x_1, x_3|x_2)$, o rozkładach brzegowych $F_{1|2}(x_1|x_2)$ oraz $F_{3|2}(x_3|x_2)$ i kopuli warunkowej $c_{13;2}$~-~czyli kopuli rozkładu $(X_1, X_3) | X_2=x_2$ i zaaplikować do niego twierdzenie Sklara \ref{thm:sklar_theorem_density}:

\begin{equation*}
	\begin{split}
		f_{3|12}(x_3|x_1,x_2)  & = \frac{f_{13|2}(x_1, x_3|x_2)}{f_{1|2}(x_1|x_2)} \\
		& = \frac{c_{13;2}(F_{1|2}(x_1|x_2), F_{3|2}(x_3|x_2); x_2)f_{1|2}(x_1|x_2)f_{3|2}(x_3|x_2)}{f_{1|2}(x_1|x_2)}\\
		& = c_{13;2}(F_{1|2}(x_1|x_2), F_{3|2}(x_3|x_2); x_2)f_{3|2}(x_3|x_2) \\
		& = c_{13;2}(F_{1|2}(x_1|x_2), F_{3|2}(x_3|x_2); x_2)c_{32}(F_3(x_3), F_2(x_2))f_2(x_2)
	\end{split}
\end{equation*}

Finalnie otrzymujemy dekompozycję $f(x_1, x_2, x_3)$ do postaci:

\begin{equation}
	\begin{split}
	f(x_1, x_2, x_3) = &c_{13;2}(F_{1|2}(x_1|x_2), F_{3|2}(x_3|x_2); x_2) \cdot \\
	& c_{23}(F_2(x_2), F_3(x_3)) \cdot c_{12}(F_1(x_1), F_2(x_2)) \cdot \\
	& f_3(x_3)f_2(x_2)f_1(x_1).
	\end{split}
	\label{eq:PCC}
\end{equation}

Powyższe czynniki są jedynie dwuwymiarowymi kopułami i (warunkowymi) rozkładami brzegowymi. Taką dekompozycję nazwiemy Pair Copula Construction (PCC). Zaletą takiej reprezentacji nad wielowymiarową kopułą jest to, że ma więcej niżej-wymiarowych komponentów, przez co pozwala na bardziej elastyczny model. Realne dane rzadko mają regularną strukturę zależności która może być dobrze opisana jedną, wielowymiarową kopułą (\cite{Czado_Vine_Copulas}), dlatego też PCC lepiej dopasowują się do danych.\\
Zwróćmy jednak uwagę na fakt, że nie jest to jedyna PCC która opisuje rozkład $f(x_1, x_2, x_3)$. Istnieją poniższe, równoważne reprezentacje:

\begin{equation}
	\begin{split}
		f(x_1, x_2, x_3) = &c_{12;3}(F_{1|3}(x_1|x_3), F_{2|3}(x_2|x_3); x_3) \cdot \\
		& c_{13}(F_1(x_1), F_3(x_3)) \cdot c_{23}(F_2(x_2), F_3(x_3)) \cdot \\
		& f_3(x_3)f_2(x_2)f_1(x_1).
	\end{split}
\end{equation}

\begin{equation*}
	\begin{split}
		f(x_1, x_2, x_3) = &c_{23;1}(F_{2|1}(x_2|x_1), F_{3|1}(x_3|x_1); x_1) \cdot \\
		& c_{13}(F_1(x_1), F_3(x_3)) \cdot c_{12}(F_1(x_1), F_2(x_2)) \cdot \\
		& f_3(x_3)f_2(x_2)f_1(x_1).
	\end{split}
\end{equation*}