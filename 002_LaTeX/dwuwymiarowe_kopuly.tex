Zgodnie z drugim filarem Basel II instytucje finansowe zobowiązane są dokonywać regularnych ćwiczeń stress testowych, w której ich kondycja finansowa poddawana jest ekstremalnym negatywnym ruchom rynkowym. Ma to na celu zbadanie kondycji finansowej w obliczu potencjalnego kryzysu ekonomicznego, oraz zbadanie czy instytucja ma zgromadzoną wystarczającą ilość kapitału aby taki kryzys przetrwać. Podczas ćwiczeń stress testowych, regulator podaje instytucjom ogólny scenariusz rynkowy (trajektorie ogólnych zmiennych makroekonomicznych), który każda instytucja musi przekuć/rozszerzyć na zbiór trajektorii zmiennych które są istotne z punktu widzenia ich biznesu. Standardową praktyką w tym procesie \emph{shock expansion} były dotychczas modele ekonometryczne, dające zazwyczaj punktowe predykcje w zadanym punkcie czasu.\\
Najnowszym trendem w stress testingu zdaje się być natomiast tzw. probabilistic stress testing (czyt. \cite{Aste_Probabilistic_Stress_Testing}), w którym ideą jest zamodelowanie systemu w taki sposób, by po zaaplikowaniu rynkowych szoków regulatora (ustaleniu niektórych komponentów wielowymiarowego losowego wektora P\&L, czy zwrotów), móc potraktować je jako zbiór warunkujący i odzyskać całe rozkłady warunkowe pozostałych komponentów systemu. Stoi to w kontraście do predykcji punktowych z obecnego podejścia ekonometrycznego. \cite{Aste_Probabilistic_Stress_Testing} w swojej pracy bada potencjalne zastosowanie rozkładów eliptycznych, wprowadzonych w rozdziale \ref{sec:rozklady_laczne}, do \emph{shock expansion}. Wyniki które otrzymuje są zgodne z oczekiwaniami: mimo, że rozkłady eliptyczne dają intuicyjne wyniki, to nie są wystarczająco elastyczne aby poprawnie zamodelować cały system. Jako tego powód, Aste wymienia narzuconą \textit{symetrię} modelu, która przejawia się w konieczności wyboru tych samych rodzin rozkładów brzegowych.\\

Powyższy przykład to jeden z długiego szeregu który ilustruje, że praktyczne problemy wymagają dowolności w specyfikacji rozkładów brzegowych. Dlatego mimo, że istnieje wiele przykładów naturalnych, poprawnych matematycznie rozszerzeń z $d=1$ do $d>2$ (jak wielowymiarowy rozkład normalny, wielowymiarowy rozkład t, wielowymiarowy rozkład gamma, etc.), to nie posiadają one dużego zastosowania w praktyce. Odpowiedzią na ten problem są funkcje kopułowe/kopuły.\\

