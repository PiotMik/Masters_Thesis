\begin{streszczenie}
	Spready grają dziś ogromną rolę na rynkach finansowych. Służą inwestorom, spekulantom i zarządzającym ryzykiem do oceny potencjalnych zysków z inwestycji, implikowania zmiennych rynkowych, czy kontrolowania ekspozycji na ryzyko. W pracy prezentujemy modele kopułowe jako narzędzia odpowiednie do statystycznej analizy współzależności między komponentami spreadu. 
	
	Literatura bogata jest w przykłady zastosowania dwuwymiarowych kopuł do modelowania dwuwymiarowych spreadów. Te modele, oparte o połączenie analizy szeregów czasowych oraz modelowania reziduów przy pomocy kopuł stały się jednym z klasycznych podejść do problemu wyceny instrumentów pochodnych na spread. Na początku lat 2000 dodatkowo rozwinęła się teoria modeli Vine Copula, pozwalających na elastyczny opis zależności wielowymiarowych. Od tamtego czasu konsekwentnie odnoszą one sukcesy w modelowaniu zjawisk w wielu dziedzinach nauki, od lotnictwa, przez biologię po finanse. Mimo tego, literatura dotycząca aplikacji Vine Copula do modelowania wielowymiarowych spreadów jest zdecydowanie ograniczona. 
	
	Praca poszerza literaturę Vine Copula o aplikację tych modeli do spreadu na 3 aktywa. Prezentujemy niezbędną teorię, oraz pokazujemy w jaki sposób zbudować symulacyjny model dla soybean crush spread, w którym numerycznie wyceniamy europejskie opcje na soybean crush spread. 
\end{streszczenie}

\begin{abstract}
	Nowadays, spreads play a vital role on global financial markets. They serve both investors, speculators and risk managers alike as a measure of potential investment turnover, to imply market variables, or as tools for controlling market risk exposures associated with combinations of risk factors. In this paper, we present copula models as a suitable tool for statistical modelling of dependency between spread components.
	
	The literature is rich with examples of bivariate copulas applied to model bivariate spreads. These models which are a combination of time series analysis and copulas have become one of the classical solutions to the problem of pricing spread derivatives. In early 2000s the copula theory was extended to Vine Copula models, allowing the flexibility of copulas to be used in higher dimensions. Since then they have been successfully applied to model phenomena in a wide range of industries: from aviation to biology to finance. Nevertheless, the literature on modelling spreads in higher dimensions using Vine Copulas is relatively limited.
	
	This paper extends the literature of Vine Copula models by presenting an application to 3-dimensional spread modelling. We present the necessary theory, and show how to build a simulation model for soybean crush spread, as well as how to numerically price European options on that asset.
	
\end{abstract}
