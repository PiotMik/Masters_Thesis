\begin{streszczenie}
	Spready grają dziś ogromną rolę na rynkach finansowych. Służą zarówno inwestorom, spekulantom jak i zarządzającym ryzykiem do oceny potencjalnych zysków z inwestycji, czy zarządzania ekspozycją na ryzyko. W pracy prezentujemy modele kopułowe jako narzędzia odpowiednie do statystycznej analizy współzależności między komponentami spreadu. 
	
	Modele wymieniane w literaturze bazują w przeważającej ilości na analizie szeregów czasowych w celu uzyskania stacjonarności danych, co pozwala następnie modelować je przy pomocy kopuł. Klasyczne podejście używa do tego jedynie komponentów spreadu i nie uwzględnienia jego poziomu. Te algorytmy postępowania są popularne wśród praktyków, ponieważ podają uniwersalne podejście do analizy spreadu, co w połączeniu z elastycznością kopuł pozwala na modelowanie szerokiej gamy spreadów, niezależnie od jego komponentów. Symulacje pokazują jednak, że takie modele nie zapewniają istotnej własności spreadu obserwowanej w praktyce jaką jest powrót do średniej.
	
	W pracy pokazujemy jak odnieść się do tego problemu poprzez dodatkowe uwzględnienie autoregresji spreadu w klasycznym podejściu kopułowym. Finalny model pozwala na symulację spreadu o pożądanych własnościach statystycznych, co badamy w pracy oraz porównujemy jego wyniki ze standardowym podejściem. 
\end{streszczenie}

\begin{abstract}
	Nowadays, spreads play a vital role on global financial markets. They serve both investors, speculators and risk managers as a gauge of potential investment turnover, or as tools to control their market risk exposure to combinations of risk factors. In this paper, we present copula models as the appropriate tool for statistical modelling of dependency between components of the spread.
	
	The fundamental idea behind such models in the literature is to first model the data as time series, extract stationary residuals, and utilize a copula to model their dependency. This classical approach typically incorporates the spreads components into the model, while leaving out the level of the spread. These algorithms are popular among practitioners since they offer a universal approach to analysis of the spread, which combined with copula flexibility allows to model a wide range of spreads regardless of the character of it's components. Simulation studies show however, that such models don't preserve an important feature of spreads, which is mean-reversion.
	
	In this paper we show how to address this problem by injecting into the classical framework an autoregressive component which aims to account for the mean-reversion of the spread. The resulting model allows to model the spread which exhibits the desired properties, which we test and compare the model's performance with the performance of the classical approach.
\end{abstract}
