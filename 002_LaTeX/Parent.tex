\documentclass[mfu, pdftex]{mgrwms}
\usepackage[utf8]{inputen
\usepackage{amsmath}
\newcommand\numberthis{\addtocounter{equation}{1}\tag{\theequation}}
\usepackage{amssymb} 
\usepackage{latexsym}
\usepackage{amsthm}
\usepackage{enumerate}
\usepackage{color}
\usepackage[polish]{babel}
\usepackage[OT4]{fontenc}
\usepackage{polski}
\allowdisplaybreaks

\begin{document}

\title{\LARGE Zastosowanie modeli kopułowych do modelowania crush spreadu.}
\author{Piotr Mikler}
\promotor{dr inż. Jerzy Dzieża}
\nralbumu{409145}
\maketitle
\slowakluczowe{crush spread, soja, opcje, copuła, vine copula}
\keywords{crush spread, soybean, option, copula, vine copula}

\newtheorem{thm}{\indent Twierdzenie}[chapter]
\newtheorem{lemma}[thm]{\indent Lemat}
\newtheorem{cor}[thm]{\indent Wniosek}
\newtheorem{obs}[thm]{\indent Obserwacja}
\newtheorem{uw}[thm]{\indent Uwaga}
\newtheorem{df}[thm]{Definicja}
\newcommand{\E}{\mathbb{E}}
\newcommand{\R}{\mathbb{R}}
\newcommand{\Pra}{\mathbb{Pra}}

\makeatletter
\newcommand*{\defeq}{\mathrel{\rlap{%
                     \raisebox{0.3ex}{$\m@th\cdot$}}%
                     \raisebox{-0.3ex}{$\m@th\cdot$}}%
                     =}
\let\c@table\c@figure
\makeatother

\tableofcontents
\begin{streszczenie}
Streszczenie po polsku, max 1 str
\end{streszczenie}

\begin{abstract}
Abstract in english, maximum of one page
\end{abstract}


\begin{wstep}    % ew. \begin{wstep}[Wprowadzenie]
Wstep

\end{wstep}

\chapter{Rozdział 1}
\mgrclosechapter

% \appendix
% \chapter{----}
%%
%-> Treść dodatku A
%%
% \mgrclosechapter
%%
%%
\bibliography{<pliki bib>} 
\end{document}

%% =========================================================== %%