Rynki finansowe pełne są przykładów spreadów pomiędzy różnymi aktywami. Od spreadów między stopami procentowymi, po różnorakie spready na rynkach energii, wiele z nich wykształciło swoją pozycję jako samoistne aktywa, nierzadko z płynnym rynkiem na instrumenty pochodne \cite{Carmona_Spread_Options}.\\
Poniżej wymienimy wybrane spready o dużej popularności, jednak w obliczu mnogości rynkowych spreadów tej listy w żaden sposób nie można nazwać wyczerpującą.\\

Spreadem nazywać będziemy różnicę pomiędzy cenami dwóch aktywów $S_1$ i $S_2$:
$$ s(t) = S_1(t) - S_2(t).$$
Jest to najprostsza wersja spreadu, jaką jest np. \emph{bid-ask} spread, czyli różnica między ceną kupna a sprzedaży aktywa na giełdzie. Definicję tę rozszerzać można jednak do liniowych kombinacji, oraz do większej liczby aktywów - co w praktyce jest nieuniknione (ze względu na różne jednostki komponentów spreadu, lub większą ilość komponentów) więc w pełnej ogólności rozważać będziemy:
$$ s(t) = \alpha_1S_1(t) - \alpha_2S_2(t) - \cdots -\alpha_nS_n(t).$$

\subsection{Rynek instrumentów o stałym dochodzie}

Rynek instrumentów o stałym dochodzie posiada bogatą gamę spreadów. Spread między stopami swapowymi różnych krajów, mierzy (implikowane) różnice pomiędzy tymi rynkami dotyczące ryzyka kredytowego i ryzyka płynności. Przykładami mogą być spready francusko-niemieckie, czy niemiecko-duńskie. Inną klasą spreadów na tym rynku jest spread kredytowy - czyli różnica między stopą zwrotu obligacji obarczonych ryzykiem, jak korporacyjne, względem zwrotu na obligacjach o znikomym ryzyku, jak rządowe. Jest to przykład \emph{quality spreadu}, czyli spreadu gdzie $S_1$ i $S_2$ są to ceny podobnego produktu, ale różniącego się jakością.\\
Inne klasy spreadów na tym rynku to \emph{tenor spreads}, jak różnica między wartościami LIBORu dla tenorów 6M i 3M, czy różnica między stopami zwrotu obligacji a bonów skarbowych. Finalnie warto wspomnieć o istnieniu \emph{issuer yield spreadów}, jak \emph{municipal-goverment} czy \emph{municipal-corporate} (\cite{Fixed_Income}).


\subsection{Rynek energii}

Przechodząc na inny rynek, zazwyczaj zmieniać się będzie charakter popularnych na nim spreadów. Rynek energii to przede wszystkim \emph{processing spready}, czyli spready wynikające z różnicy cen zakupu surowca, a sprzedaży wytworzonego produktu, oraz \emph{temporal spready}/\emph{calendar spready} - czyli różnice w cenie energii w różnych porach dnia/różnych porach roku.\\
Na tym rynku, spready pozwalają oszacować koszty produkcji, transportu, przechowania towaru, czy efektywność linii produkcyjnej. Do najpilniej obserwowanych z nich należy \emph{crack spread}, czyli różnica między ceną ropy naftowej a wyprodukowanych z niej produktów ropopochodnych. Podobnego rodzaju spreadów, o równie istotnym znaczeniu jest wiele - między innymi \emph{dark spread}, \emph{spark spread}, \emph{quark spread} i inne. Każdy odnosi się do różnicy ceny innego surowca (gazu, węgla, uranu i innych) a ceny energii z niego wyprodukowanej.\\
Wraz ze wzrostem świadomości społeczeństwa na temat zmian klimatycznych, oraz wzrostem regulacji mających na celu zmniejszenie ilości wytwarzanych gazów cieplarnianych, pojawiły się również ciekawe rodzaje spreadów jak \emph{clean dark spread}, czy \emph{clean spark spread}. Są one wariacjami powyższych spreadów energetycznych, z dodatkowym uwzględnieniem trzeciego komponentu, jakim jest konieczność nabycia certyfikatów emisyjnych (\cite{Carmona_Clean_Spreads}).

\subsection{Rynek surowców}

Pod hasłem spreadów na rynku surowców w dużej mierze kryją się processing spready wymienione wcześniej w ramach rynku energii. Poza tym jednak istotną rolę pełnią tu \emph{forward spready} - czyli różnice w cenach kontraktów forward na surowce oraz \emph{locational spready} - różnice między cenami tego samego surowca w dwóch różnych miejscach na świecie. Pełnią one rolę informacyjną, ponieważ można z nich implikować sentyment uczestników rynku dotyczący przyszłych cen surowców i ich substytutów (\emph{corn-oat} czy \emph{wheat-corn} spread), oraz podobnie jak w przypadku rynku energii - efektywność linii produkcyjnej.\\
Przykładem processing spreadu na rynku surowców będzie \emph{soybean crush spread}, dokładniej omówiony w rozdziale \ref{ch:zastosowanie_do_danych}, polegający na kupnie ziaren soi a sprzedaży olejku sojowego oraz mączki sojowej - realizowany jest najczęściej w formie forward spreadu (\cite{Agro_Spreads}).