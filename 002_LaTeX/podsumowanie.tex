\begin{wstep}[Wnioski]    % ew. \begin{wstep}[Wprowadzenie]
	W pracy omówiliśmy model ARIMA-GARCH-VineCopula w kontekście symulacji spreadu wielu aktywów. Podaliśmy teorię kopuł, oraz struktur Vine Copula z wyszczególnieniem ich zastosowań w praktyce, oraz omówiliśmy ich związek z rodzajami struktur zależności alternatywnymi dla najprostszej liniowej korelację Pearsona.\\
	Przedstawiliśmy wyniki kalibracji modelu do tygodniowych cen zamknięcia komponentów \emph{soybean crush spreadu}. W trakcie analizy udowodniliśmy ciężkoogonowość logzwrotów, oraz brak istotnej struktury autokorelacji czy częściowej autokorelacji w komponentach spreadu. Dane przejawiają heteroskedastyczność, dającą się skutecznie zamodelować przy pomocy modelu GARCH(2, 3).\\
	Zbadana została struktura zależności reziduów modeli GARCH, do których dopasowany został model Vine Copula bazujący na kopułach: Gaussowskiej i Franka. Model ukazał istotność soi jako wiodącego szeregu czasowego o największym wpływie na pozostałe komponenty, oraz wskazał na brak istotnej zależności w ogonach reziduów co przejawiło się w dopasowaniu kopuł o braku współczynnika zależności ogonów.\\
	Porównaliśmy model Vine Copula z modelem wielowymiarowej kopuły gaussowskiej dochodząc do wniosku, że oba modele uchwycają ciężkoogonowość spreadu, oraz dają podobne rozkłady logzwrotów dla większości kwantyli - jednak Vine Copula przejawia cięższe ogony w kwantylach ekstremalnych ($0.01$, $0.99$).\\
	Finalny symulacyjny model potrafi produkować realistyczne realizacje trajektorii spreadu, co pozwala wykorzystywać go do wyceny instrumentów pochodnych na spread metodą Monte Carlo - co zaprezentowane zostało dla przypadku europejskich opcji.
	
\end{wstep}
